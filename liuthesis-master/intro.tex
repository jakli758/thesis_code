%%% Intro.tex --- 
%% 
%% Filename: Intro.tex
%% Description: 
%% Author: Ola Leifler
%% Maintainer: 
%% Created: Thu Oct 14 12:54:47 2010 (CEST)
%% Version: $Id$
%% Version: 
%% Last-Updated: Thu May 19 14:12:31 2016 (+0200)
%%           By: Ola Leifler
%%     Update #: 5
%% URL: 
%% Keywords: 
%% Compatibility: 
%% 
%%%%%%%%%%%%%%%%%%%%%%%%%%%%%%%%%%%%%%%%%%%%%%%%%%%%%%%%%%%%%%%%%%%%%%
%% 
%%% Commentary: 
%% 
%% 
%% 
%%%%%%%%%%%%%%%%%%%%%%%%%%%%%%%%%%%%%%%%%%%%%%%%%%%%%%%%%%%%%%%%%%%%%%
%% 
%%% Change log:
%% 
%% 
%% RCS $Log$
%%%%%%%%%%%%%%%%%%%%%%%%%%%%%%%%%%%%%%%%%%%%%%%%%%%%%%%%%%%%%%%%%%%%%%
%% 
%%% Code:


\chapter{Introduction}
\label{cha:introduction}

\section{Motivation}
\label{sec:motivation}

Atypical femoral fractures (AFFs) are a rare type of fractures in the femur. Unlike normal femoral fractures (NFFs), 
they are not caused by high immediate stress, but are related to long-term bisphosphonate medication prescribed to patients 
with osteoporosis \cite{schilcher_bisphosphonate_2011}. The identification of AFFs remains challenging in clinical practice, 
and these fractures are frequently overlooked unless explicitly assessed. Less than 7\% of all cases are correctly 
diagnosed in radiology reports \cite{zdolsek_deep_2021}. In precision medicine and personalized healthcare, accurate diagnosis is
crucial for effective treatment. AFFs have a high rate of reopertion \cite{bogl_increased_2020} and therefore require a special focus.   
To avoid unecessary surgeries, reliable and early detection is essential.

To support medical doctors in improving diagnostic accuracy, deep neural networks have been trained to classify AFFs and 
NFFs \cite{zdolsek_deep_2021}. However, the dataset used in previous studies was relatively small, consisting of approximately 
1,000 images from around 375 patients. In this thesis, we utilize a larger dataset comprising approximately
4300 images from around 1200 patients. Nevertheless, this dataset presents a common challenge in medical imaging: severe class imbalance. 
The minority class (AFF) is significantly underrepresented, a well known issue in machine learning, that can negatively affect model 
performance \cite{ghavidel_machine_2025}. 
Several strategies have been proposed to address class imbalance. Oversampling increases the number of minority class samples, 
often by duplicating existing images, while undersampling reduces the number of majority class samples. Although these techniques can 
improve performance, they also introduce potential drawbacks, such as overfitting in the case of oversampling and information loss in 
the case of undersampling.

With the rapid development of generative models, alternative approaches for dataset augmentation have emerged. Instead of reusing 
existing images, synthetic samples can be generated by learning the underlying high-dimensional data distribution.

Over the past decade, generative adversarial networks (GANs) have been the dominant architecture for generative tasks 
in medical imaging \cite{ali_generative_2025}. 
More recently, different versions of diffusion models showed remarkable preformance in various areas and
have increasingly been applied to medical imaging tasks \cite{shi_diffusion_2025}. 

Generative models can be conditioned in different ways during the synthesis process. Previous thesis work used noise as input to generate
synthetic images \cite{wang_synthetic_2025} in so-called noise-to-image processing. 
This thesis focuses on image-to-image translation, where the model learns to translate images from one domain to another. This allows a more 
controlled synthesis conditioned on existing anatomical structures.
A well established architecture for this task is CycleGAN \cite{zhu_unpaired_2018}. We compare this approach with a diffusion-based model 
optimized for image-to-image translation, CycleDiff \cite{zou_cyclediff_2026}.

\section{Research questions}
\label{sec:research-questions}

The aim of this thesis is to establish whether synthesising data using generative models can improve the classification of rare fracture types. 
For this purpose, a CycleGAN and a CycleDiff model are trained on X-rays of AFFs and NFFs to create balanced datasets. The generative models
are evaluated based on how well the generated images imitate the real data.

To assess whether generative augmentation improves fracture classification, a ResNet-50 classifier is trained under different conditions. 
First, a baseline model is trained using only the real data. Second, a classifier is trained using real data augmented with synthetically
generated images. The performance of these models is then compared using standard metrics like accuracy and the F1-score and ,additionally, using 
statistical tests.

Furthermore, to determine whether generative augmentation provides an advantage over conventional data augmentation techniques 
(such as rotation, flipping, and cropping), a ResNet-50 trained on real data augmented with standard methods is compared to a 
ResNet-50 trained on real data augmented with generative models.

\begin{enumerate}
\item Does CycleGAN or CycleDiff perform better in the task of generating synthetic
images by transforming between AFF and NFF images?

\item Does a ResNet-50 classifier trained on real data augmented with generated images perform significantly 
better than a classifier trained only on real data?

\item Does generative augmentation lead to significantly better classification performance compared 
to the standard augmentation techniques rotation, flipping and cropping \todo{or oversampling instead?}?

\end{enumerate}


\section{Delimitations}
\label{sec:delimitations}

\todo{tbd}
This is where the main delimitations are described. For
example, this could be that one has focused the study on a
specific application domain or target user group. In the
normal case, the delimitations need not be justified.

%\nocite{scigen}
%We have included Paper \ref{art:scigen}

%%%%%%%%%%%%%%%%%%%%%%%%%%%%%%%%%%%%%%%%%%%%%%%%%%%%%%%%%%%%%%%%%%%%%%
%%% Intro.tex ends here


%%% Local Variables: 
%%% mode: latex
%%% TeX-master: "demothesis"
%%% End: 
